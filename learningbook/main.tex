\documentclass[a4paper, twoside, openany]{book}

% Подключение пакетов                                                                                                                                                                              
\usepackage[utf8]{inputenc}                       % Изменение кодировки ан UTF8 и как следствие поддержка кириллицы
\usepackage[russian]{babel}                       % Основноый язык - русский, библиотека для проверки орфографии и корректных переносов
\usepackage{fontspec}                             % Пакет для работы с системными шрифтами                                                                                                         
\usepackage[paperheight=297mm,  paperwidth=210mm,       top=20mm,       bottom=15mm,    left=20mm,
        right=10mm]{geometry}                     % Параметры страницы                                                                                                                             
\usepackage{amsmath}                              % Мат формулы                                                                                                                                    
\usepackage{graphicx}                             % Пакет для вставки изображений                                                                                                                  
\usepackage{hyperref}                             % Для создания ссылок
\usepackage{listings}                             % Для вставки кода
\usepackage{xcolor}                               % Для цветовой схемы
\usepackage{listingsutf8}                         % Поддержка UTF-8 в listings

% Установка основного шрифта. ТаймсНюРоман                                                           
\setmainfont{Times New Roman}
% Установка моноширного шрифта для кода
\setmonofont{Liberation Mono}

% Цвета для оформления
\definecolor{codeblue}{rgb}{0.0, 0.0, 0.5}
\definecolor{codegray}{rgb}{0.5,0.5,0.5}
\definecolor{codeorange}{rgb}{1.0, 0.65, 0.0} % Определяем оранжевый цвет
\definecolor{backcolour}{rgb}{0.95,0.95,0.92}

\lstdefinelanguage{Zig}{
    keywords={const, var, fn, pub, return, if, else, while, for, switch, break, continue, defer, errdefer, try, catch, async, await, suspend, resume, comptime, noalias, inline, noinline, export, extern, packed, struct, enum, union, error, anyerror, undefined, null, true, false},
    sensitive=true,
    morecomment=[l]{//},
    morestring=[b]",
    morestring=[b]'
}

\lstset{
    language=Zig,
    frame=single, % Рамка вокруг кода
    numbers=left, % Нумерация строк слева
    numberstyle=\color{gray}, % Стиль нумерации
    keywordstyle=\color{codeorange}, % Цвет ключевых слов
    commentstyle=\color{codeblue}, % Цвет комментариев
    stringstyle=\color{red}, % Цвет строк
    captionpos=b, % Позиция заголовка (b - снизу)
    basicstyle=\ttfamily\small, % Моноширный шрифт
    backgroundcolor=\color{backcolour}, % Цвет фона
    captionpos=b, % Позиция заголовка (b - снизу)
    inputencoding=utf8, % Поддержка UTF-8
}

\title{От языка человеческого к языку машинному}
%Моя книга, как хочу - так называю
\author{Nnu Noiiiy Octops}
\date{Однажды эта книга увидет свет...}

\begin{document}

\maketitle

\tableofcontents

\chapter{Введение}
 \fontsize{14pt}{14pt}\selectfont
 Язык программирования Zig — это современный язык, который предлагает множество возможностей
 для разработчиков. В этой книге мы рассмотрим основные концепции и возможности Zig.

\section{Установка}
 \fontsize{14pt}{14pt}\selectfont
 Чтобы начать работать с Zig, вам нужно установить его. Подробности можно найти на
 \href{https://ziglang.org/download/}{официальном сайте Zig}.

\section{Первый пример}
 \fontsize{14pt}{14pt}\selectfont
Давайте напишем наш первый пример на Zig. Мы создадим простую программу, которая выводит "Hello, World!".

\subsection{Код}
 \fontsize{14pt}{14pt}\selectfont
Вот как выглядит код:

\begin{lstlisting}[language=Zig, caption={Пример на Zig}]
const std = @import("std");

pub fn main() void {
    std.debug.print("Hello, World!\n", .{});
}
\end{lstlisting}

\section{Основные конструкции}
 \fontsize{14pt}{14pt}\selectfont
В этом разделе мы рассмотрим основные конструкции языка Zig.

\subsection{Переменные}
 \fontsize{14pt}{14pt}\selectfont
Переменные в Zig объявляются с помощью ключевого слова \texttt{var} или \texttt{const}. Например:

\begin{lstlisting}[language=Zig, caption={Пример на Zig}]
const x = 5; // constant
var y = 10;   //  \textcolor{codeblue}{синий текст}
\end{lstlisting}

\section{Ссылки на разделы}
 \fontsize{14pt}{14pt}\selectfont
Вы можете вернуться к разделу \ref{Установка} для получения информации о том, как установить Zig.

\chapter{Заключение}
 \fontsize{14pt}{14pt}\selectfont
В этой книге мы рассмотрели ничегооо 

\end{document}
