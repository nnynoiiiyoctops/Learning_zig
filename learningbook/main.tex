\documentclass[a4paper, twoside, openany]{book}

% Подключение пакетов                                                                                                                                                                              
\usepackage[utf8]{inputenc}                        % Изменение кодировки ан UTF8 и как следствие поддержка кириллицы
\usepackage[russian]{babel}                        % Основноый язык - русский, библиотека для проверки орфографии и корректных переносов
\usepackage{fontspec}                             % Пакет для работы с системными шрифтами                                                                                                         
\usepackage[paperheight=297mm,  paperwidth=210mm,       top=20mm,       bottom=15mm,    left=20mm,
        right=10mm]{geometry}                     % Параметры страницы                                                                                                                             
\usepackage{amsmath}                              % Мат формулы                                                                                                                                    
\usepackage{graphicx}                             % Пакет для вставки изображений                                                                                                                  
\usepackage{hyperref}                             % Для создания ссылок
\usepackage{listings}                             % Для вставки кода
\usepackage{xcolor}                               % Для цветовой схемы

% Установка основного шрифта. ТаймсНюРоман                                                           
\setmainfont{Times New Roman}
% Установка моноширного шрифта для кода
\setmonofont{Liberation Mono}
\lstdefinelanguage{Zig}{
    keywords={const, var, pub, fn, return, if, else, while, for, break, continue, switch, case, defer, try, catch, error, true, false, null},
    sensitive=true,
    comment=[l]{//},
    morecomment=[s]{/*}{*/},
    string=[b]"",
    morestring=[b]'',
}

% Настройка подсветки кода
\lstset{
    language=Zig,
    backgroundcolor=\color{white},
    basicstyle=\ttfamily\small, % Уменьшенный шрифт для кода
    keywordstyle=\color{blue}\bfseries, % Жирный синий для ключевых слов
    commentstyle=\color{green}, % Зеленый для комментариев
    stringstyle=\color{red}, % Красный для строк
    showstringspaces=false,
    numbers=left,
    numberstyle=\tiny\color{gray},
    stepnumber=1,
    numbersep=5pt,
    tabsize=2,
    breaklines=true,
    frame=single, % Рамка вокруг кода
}

\title{Потом придумаю}
\author{Nnu Noiiiy Octops}
\date{\today}


\begin{document}

\maketitle

\tableofcontents

\chapter{Введение}
 \fontsize{14pt}{14pt}\selectfont
Язык программирования Zig — это современный язык, который предлагает множество возможностей для разработчиков. В этой книге мы рассмотрим основные концепции и возможности Zig.

\section{Установка}
 \fontsize{14pt}{14pt}\selectfont
Чтобы начать работать с Zig, вам нужно установить его. Подробности можно найти на \href{https://ziglang.org/download/}{официальном сайте Zig}.

\section{Первый пример}
 \fontsize{14pt}{14pt}\selectfont
Давайте напишем наш первый пример на Zig. Мы создадим простую программу, которая выводит "Hello, World!".

\subsection{Код}
 \fontsize{14pt}{14pt}\selectfont
Вот как выглядит код:

\begin{lstlisting}
const std = @import("std");

pub fn main() !void {
    try std.debug.print("Hello, World!\n", .{});
}
\end{lstlisting}

\section{Основные конструкции}
 \fontsize{14pt}{14pt}\selectfont
В этом разделе мы рассмотрим основные конструкции языка Zig.

\subsection{Переменные}
 \fontsize{14pt}{14pt}\selectfont
Переменные в Zig объявляются с помощью ключевого слова \texttt{var} или \texttt{const}. Например:

\begin{lstlisting}
const x = 5; // Константа
var y = 10;   // Переменная
\end{lstlisting}

\section{Ссылки на разделы}
 \fontsize{14pt}{14pt}\selectfont
Вы можете вернуться к разделу \ref{Установка} для получения информации о том, как установить Zig.

\chapter{Заключение}
 \fontsize{14pt}{14pt}\selectfont
В этой книге мы рассмотрели ничегооо 

\end{document}
